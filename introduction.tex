\section{Introduction} \label{sec:intro}

The Total Knowledge of I/O (TOKIO) framework~\cite{Lockwood2017} connects data from component-level monitoring tools across the I/O subsystems of HPC systems.  Rather than build a universal monitoring solution and deploy a scalable data store to retain all monitoring data, TOKIO connects to \emph{existing} best-in-class monitoring tools and databases, indexes these tools' data, and presents the data from multiple connectors in a single, coherent view to downstream analysis tools and user interfaces.

At a high level, we propose three basic axioms of characterizing storage and I/O subsystems today:

\begin{itemize}[leftmargin=*]
\item \textbf{I/O systems are complex systems that fail in complex ways.}
High-performance parallel I/O is made possible by increasingly complex subsystems which attempt to satisfy three contradictory goals:
(1) delivering high bandwidth and low latency in a scalable fashion,
(2) providing durability and persistence of data, and
(3) enabling as much POSIX compatibility as possible to enable portability.
Being complex systems, storage systems also fail in complex ways;
while an outright outage may be straightforward to diagnose, the more complex failure modes, such as fail-slow~\cite{Gunawi2018}, require commensurately more complex diagnosis procedures.

\item \textbf{Complex architectures beget a complex set of essential tools.}
Storage systems are composed of software, middleware, and hardware created by different a variety of different technology providers who are the world's experts in the components they provide.
As a result, they are also best qualified to provide the tools that report on the performance and well-being of those components.

\item \textbf{Broad expertise is required to gain insight from these tools.}
Drawing insight from a diversity of tools that operate on complex systems requires a combination of expert knowledge of the entire I/O subsystem and well-defined and composable analysis methods.
\end{itemize}

These three axioms speak to the need for I/O characterization frameworks that take a holistic approach to performance analysis and capture the full resolution of the available telemetric data on I/O subsystems while simultaneously providing simpler, semantically relevant access to those data.
Such a framework would be built upon the following design criteria, motivated by the three axioms of I/O characterization frameworks:

\begin{enumerate}[leftmargin=*]
\item \label{design:existingtools} \textbf{Use existing tools already in production.}
A variety of tools already exist to characterize individual components of the I/O subsystem, and different HPC centers often already have many of these tools in production.
Using existing tools not only allows us to leverage the component-specific expertise of the individuals who created each tool, but it reduces the burden of integrating the framework with HPC centers since it builds upon the tools with which facilities staff are already familiar.

\item \label{design:leavedata} \textbf{Leave data where it is.}
Attempting to aggregate all component-level telemetric data in a centralized data warehouse often requires coercing the output data to fit a schema, and this process can be lossy.
Furthermore, the overheads of maintaining a data warehouse suitable for aggregating all data can be high if one is not already deployed at an HPC facility.
Thus, the framework should meet the tools where they are and work with data as it is natively generated. 
Organizing and querying the data can be achieved by indexing the different data types and data sources rather than replicating and normalizing them.

\item \label{design:accessible} \textbf{Make data as accessible as possible.}
The principal role of the framework is to provide semantically sensible and consistent access to the diversity of data generated by component-level tools.
Users should be able to request a single logical quantity (such as bytes written to a storage server) and be given the data in a standard data format without having to understand the tool that collected that data.
Furthermore, the framework should not require escalated privileges to be useful; access controls are better handled by the tools and data sources that the framework indexes.
\end{enumerate}

In previous work, we presented the Total Knowledge of I/O (TOKIO) framework as a formalization of these requirements on a conceptual basis and demonstrated the new insights into I/O performance that such an approach enables~\cite{Lockwood2017}.
In this work, we present \emph{pytokio}, a reference implementation of the TOKIO framework implemented  in Python 2, which is available as a freely downloadable library.
We also present accompanying management and analysis tools that the community can use to extract meaningful and holistic insights from the data that is already produced on Cray XC environments.