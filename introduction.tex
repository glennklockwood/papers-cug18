\section{Introduction}

The Total Knowledge of I/O (TOKIO) framework~\cite{Lockwood2017} connects data from component-level monitoring tools across the I/O subsystems of HPC systems.  Rather than build a universal monitoring solution and deploy a scalable data store to retain all monitoring data, TOKIO connects to \emph{existing} best-in-class monitoring tools and databases, indexes these tools' data, and presents the data from multiple connectors in a single, coherent view to downstream analysis tools and user interfaces.

At a high level, we propose three basic axioms of characterizing storage and I/O subsystems today:

\begin{enumerate}[leftmargin=*]
\item \textbf{I/O systems are complex systems that fail in complex ways.}
High-performance parallel I/O is made possible by increasingly complex subsystems which attempt to satisfy three contradictory goals:
(1) delivering high bandwidth and low latency in a scalable fashion,
(2) providing durability and persistence of data, and
(3) enabling as much POSIX compatibility as possible to enable portability.
Being complex systems, storage systems also fail in complex ways;
while an outright outrage may be straightforward to diagnose, the more complex failure modes, such as fail-slow~\cite{Gunawi2018}, require commensurately more complex diagnosis procedures.

\item \textbf{Complex architectures beget a complex set of essential tools.}
Storage systems are composed of software, middleware, and hardware created by different a variety of different technology providers who are the world's experts in the components they provide.
As a result, they are also best qualified to provide the tools that report on the performance and well-being of those components.

\item \textbf{Broad expertise is required to gain insight from these tools.}
Drawing insight from a diversity of tools that operate on complex systems requires a combination of expert knowledge of the entire I/O subsystem and well-defined and composable analysis methods.
\end{enumerate}

These three axioms speak to the need for I/O characterization frameworks that take a holistic approach to performance analysis and capture the full resolution of the available telemetric data on I/O subsystems while simultaneously providing simpler, semantically relevant access to those data.
Such a framework would be built upon the following design criteria which are motivated by the three axioms of I/O characterization frameworks:

\begin{itemize}[leftmargin=*]
\item use the existing best-in-class tools since they know best

\item leave data where it is - don't convert to lossy non-standard formats unnecessarily, and don't set up elaborate data warehouses

\item the role of the framework is to provide semantic sensibility and consistency across the diversity of tools and data formats
\end{itemize}

In previous work, we presented the Total Knowledge of I/O (TOKIO) framework as a formalization of these requirements on a conceptual basis and demonstrated the new insights into I/O performance that such an approach enables~\cite{Lockwood2017}.
In this work, we present \emph{pytokio}, a reference implementation of the TOKIO framework implemented  in Python 2, which is available as a freely downloadable library.
We also present accompanying management and analysis tools that the community can use to extract meaningful and holistic insights from the data that is already produced on Cray XC environments.